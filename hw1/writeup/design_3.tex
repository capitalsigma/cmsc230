\documentclass{article}
\usepackage{amsmath}


\newcommand{\field}[1]{\item \texttt{ #1 }}

\begin{document}
\title{Homework 1 - Design Writeup}
\author{Patrick Collins}
\date{\today}
\maketitle

\section{graph.c} 
File-level constants:
\begin{itemize}
\item Error codes:
  \begin{itemize}
  \field{INVALID\_INPUT\_ERROR}\\
    Used for graphs with invalid distances. Violations of Graph
    invariant \#1.
  \field{MALFORMED\_INPUT\_ERROR}\\
    Used for files with the wrong number of lines, etc. Violations of Graph
    invariant \#2. 
  \end{itemize}
\item Values:
  \begin{itemize}
  \field{INF\_DIST 1000}
  \field{MAX\_DIST 10000000}
  \end{itemize}
\end{itemize}

\subsection{Graph}

\subsubsection{Description}
The implementation here closely follows the Wikipedia description of
the algorithm linked in the assignment. The description is omitted.

\subsubsection{Implementation Overview}
Properties:
\begin{itemize}
  \field{int verticies}
  \field{int** matrix}\\
\end{itemize}
Methods:
(Following the Python sytle of a leading underscore for private
methods, providing some degree of information hiding without
unnecessarily complicating testing. This convention is followed for
the remainder of the document.)
\begin{itemize}
  \field{int \_get\_element(Graph* self, int row, int column)}
  \field{void \_set\_element(Graph* self, int row, int column, int
    val)}
  \field{void \_update\_dist(Graph* self, int i, int j, int k)}
  \field{void \_update\_dists(Graph* self, int k)}
  \field{bool graph\_eq(Graph* g1, Graph* g2)}
  \field{int get\_verticies(Graph* self)}
  \field{void solve\_graph(Graph* self)}
  \field{void print\_graph(Graph* self)}\\
\end{itemize}
Constructors:
\begin{itemize}
  \field{Graph* graph\_from\_file(file *f)}\\
\end{itemize}
Destructors:
\begin{itemize}
  \field{void free\_graph(Graph* g)}\\
\end{itemize}
Invariants: 
\begin{enumerate}
\item For each element \texttt{e} $\in$ \texttt{matrix}, \texttt{e <
    MAX\_DIST || e == INF\_DIST}
\item \texttt{rows == columns == verticies}
\end{enumerate}

\section{tsgraph.c}
\subsection{\_DistArgs}
Supporting module for \texttt{TSGraph}.\smallskip\\ 
Properties:
\begin{itemize}
  \field{TSGraph* tsgraph}
  \field{int i\_start}
  \field{int i\_end}
  \field{int k}
  \field{int tid}\\
\end{itemize}
Constructor:
\begin{itemize}
  \field{\_DistArgs* \_make\_dist\_args(TSGraph* tsgraph, int i\_start, int i\_end,
    int k, int tid)}\\
\end{itemize}
Destructor:
\begin{itemize}
  \field{\_free\_dist\_args(\_DistArgs* args)}
\end{itemize}

\subsection{TSGraph}
\subsubsection{Implementation Overview}
Properties:
\begin{itemize}
  \field{Graph* graph}
  \field{int num\_threads}
  \field{int k}
  \field{volatile bool[] working\_threads}\\
\end{itemize}
Methods:
\begin{itemize}
  \field{Graph* get\_graph(TSGraph* self)}
  \field{TSGraph* \_update\_dist\_p(TSGraph* self, int i\_start, int i\_end, int
    k, int tid)}
  \field{void* \_update\_dist\_p\_wrapper(\_DistArgs* args)}
  \field{TSGraph* \_update\_dists\_p(TSGraph* self)}
  \field{bool still\_working(TSGraph* self)}
  \field{int get\_k(TSGraph* self)}\\
\end{itemize}
Constructor:
\begin{itemize}
  \field{TSGraph* tsgraph\_from\_graph(Graph* graph, int num\_threads)}\\
\end{itemize}
Destructor:
\begin{itemize}
  \field{void free\_tsgraph(TSGraph* self)}\\
\end{itemize}
Invariants:
\begin{enumerate}
\item No two threads will call \texttt{\_update\_dist\_p} at the same
  time unless the argument \texttt{k} is equal in each.
% \item No two threads will call \texttt{\_update\_dist\_p} at the same
%   time with overlapping values for \texttt{i\_start} and \texttt{i\_end}.
\end{enumerate}

\subsubsection{Description}
Noting the description of the algorithm in terms of its recursive
formula (as listed on Wikipedia):
\begin{align*}
  \texttt{shortestPath}(i, j, 0) & =  w(i, j)\\
  \texttt{shortestPath}(i, j, k+1) & = 
  \texttt{min}(\texttt{shortestPath}(i, j, k), 
  \texttt{shortestPath}(i, k+1, k), \\ 
&\texttt{shortestPath}(k+1, j, k))
\end{align*}
\noindent
it is evident that successive evaluations of \texttt{shortestPath} are
independent for values of $i$ and $j$ but not $k$. Therefore, the
\texttt{ThreadsafeGraph} will aim to guarantee that every thread is
evaluating \texttt{shortestPath} for different values of $i$ and $j$
but the same value of $k$. 

This will be implemented by ``chunking'' out the work on $i$, the
outer loop index, and using a \texttt{bool} array,
\texttt{working\_threads}, to signal when all threads have completed
work. Specifically, for $n$ threads, \texttt{working\_threads} will be
initialized as an $n$-element array containing only
\texttt{true}. When the $i$th thread completes its work, it will set
the $i$th element of \texttt{working\_threads} to \texttt{false}. When
every element of \texttt{working\_threads} becomes \texttt{false}, \texttt{self->k} will be incremented, every element of \texttt{working\_threads} will be reset to
\texttt{true}, the next
loop iteration will continue, until \texttt{self->k == self->graph->vertices}.

Based on an implementation of \texttt{\_update\_dist} which will take
constant time, \texttt{\_update\_dist\_p} will take time that is linear
on the difference between \texttt{i\_min} and \texttt{i\_max}. For
each thread except for the last, 
\begin{align*}
\texttt{i\_max - i\_min ==
  self->graph->verticies / self->num\_threads}.
\end{align*}
The last thread will
be within \texttt{self->graph->vertices \%  self->num\_threads} of
this value. Therefore, work will be divided equally between threads,
providing for a good load imbalance.

Using an array of \texttt{bool} values for synchronization will ensure
that, for every $i$ and $j$, \texttt{shortestPath}$(i, j, k+1)$ is
evaluated after \texttt{shortestPath}$(i, j, k)$, preventing a data
race. Additionally, it avoids starting and joining
\texttt{self->num\_threads} threads per loop iteration, lowering
overhead. 

\newpage

\section{Test Plan}
\subsection{Graph}
Hypothesis: Graphs are instantiated correctly.\\
Tests:
\begin{itemize}
\item Every graph is equal to itself.
\item The value of each vertex of a newly-initialized graph is equal
  to the corresponding value in the file it is contained in.
\item \texttt{INVALID\_INPUT\_ERROR} is raised for graphs with
  impossible distances.
\item \texttt{MALFORMED\_INPUT\_ERROR} is raised for graphs that do
  not meet format specifications.\\
\end{itemize}
Hypothesis: Graphs are solved correctly.\\
Tests:
\begin{itemize}
\item Nodes with no outgoing edges remain \texttt{INF\_DIST} from all
  other nodes after a graph is solved.
\item 5 nodes in a horizontal line a unit distance away from each
  other, bidirectionally, are correctly reported as being 0, 1, 2, 3 and 4 units away
  from the rightmost node, and vice-versa for the leftmost.
\item 5 nodes in a vertical line a unit distance away from each
  other, bidirectionally, are correctly reported as being 0, 1, 2, 3 and 4 units away
  from the bottommost node, and vice-versa for the topmost.
\item 5 nodes in a horizontal line a unit distance away from each
  other, bidirectionally, with a 2-unit long path from the leftmost to the rightmost
  node, are correctly reported as being 0, 1, 2, 3, 3 and 2 units away
  from each other.
\item An undirected graph with two nodes \texttt{MAX\_DIST - 1} away from
  a third undergoes no change when solved.
\end{itemize}

\subsection{TSGraph}
Hypothesis: No two threads will evaluate \texttt{shortestPath}$(i, j,
k)$ for two different values of $k$ at the same time. Furthermore, $k$
is evaluated for each $i, j$ exactly once.\\
Tests:
\begin{itemize}
\item Given a \texttt{TSGraph*} object \texttt{tsg} with
  \texttt{tsg->num\_threads = n} and \texttt{tsg->k = 0}, for $i$ = 0 to $n - 2$, evaluate
  \texttt{\_update\_dist\_p(tsg, 0, 0, i)}. Then:
  \begin{itemize}
  \item \texttt{still\_working(tsg) == true}.
  \item \texttt{get\_k(tsg) == 0}.
  \end{itemize}
\item After, evaluate \texttt{\_update\_dist\_p(tsg, 0, 0, n - 1)}. Then:  
  \begin{itemize}
  \item \texttt{still\_working(tsg) == true}.
  \item \texttt{get\_k(tsg) == 1}.
  \end{itemize}
\item Repeat until \texttt{k == tsg->graph->verticies}, which will
  occur in \texttt{k - 1} steps.\\
\end{itemize}
Hypothesis: \texttt{TSGraph} objects are solved correctly.\\
Tests:
\begin{itemize}
\item Solving the same testcases as listed under the Graph tests
  yields the same graph as in that case, for 1, 2, 3, 4 and 5 threads.
\end{itemize}


\end{document}
