\documentclass{article}
\usepackage{amsmath}

\newcommand{\contradiction}{{\hbox{%
    \setbox0=\hbox{$\mkern-3mu\times\mkern-3mu$}%
    \setbox1=\hbox to0pt{\hss$\times$\hss}%
    \copy0\raisebox{0.5\wd0}{\copy1}\raisebox{-0.5\wd0}{\box1}\box0
}}}

\usepackage{listings}
\usepackage{color}

\definecolor{dkgreen}{rgb}{0,0.6,0}
\definecolor{gray}{rgb}{0.5,0.5,0.5}
\definecolor{mauve}{rgb}{0.58,0,0.82}

\lstset{frame=tb,
  tabsize=3,
  language=Java,
  aboveskip=3mm,
  belowskip=3mm,
  showstringspaces=false,
  columns=flexible,
  basicstyle={\small\ttfamily},
  numbers=none,
  numberstyle=\tiny\color{gray},
  keywordstyle=\color{blue},
  commentstyle=\color{dkgreen},
  stringstyle=\color{mauve},
  breaklines=true,
  breakatwhitespace=true
}


\newcommand{\field}[1]{\item \texttt{#1}}
\newcommand{\num}[1]{\setcounter{enumi}{#1}}

\begin{document}
\title{Homework 1 - Theory}
\author{Patrick Collins}
\date{\today}
\maketitle

\begin{enumerate}
\setcounter{enumi}{1}
% 2, 3, 4, 5, 7, 8, 11, 14, 15, 16
% safety -- bad thing never happens
% liveness -- good things eventually happen
\item Identify safety and liveness properties. 
  \begin{enumerate}
  \item Liveness. The ``good thing'' is being served.
  \item Liveness. The ``good thing'' is receiving the ball that has
    beeen thrown.
  \item Safety. The ``bad thing'' is waiting for longer than it takes
    for the first thread to exit its critical section.
  \item Safety. The ``bad thing'' is waiting more than one second to
    print a message.
  \item Liveness. The ``good thing'' is priting a message.
  \item Safety. The ``bad thing'' is the cost of living decreasing.
  \item Liveness. The ``good thing'' is taxes being paid and resources
    being freed by death.
  \item Safety. The ``bad thing'' is to be identified as something
    other than a Harvard man.
  \end{enumerate}

\item Design a producer-consumer protocol with interrupts. \\
Alice and Bob each have a can on their windowsill. Alice does the following:
\begin{enumerate}
\item She waits until her can is down.
\item She releases the pets.
\item When the pets return, Alice checks whether they finished the
  food. If they did, she resets her can and knocks over Bob's.\\
\end{enumerate}
Bob does the following:
\begin{enumerate}
\item He waits until his can is down.
\item He puts food in the yard.
\item He resets his can.
\item He pulls the string and knocks Alice's can down.
\end{enumerate}

\item Design a winning strategy, when...
\begin{itemize}
\item You know the initial state of the switch is off.\\
Designate one prisoner as a counter, who thinks of a number $x = 0$. Whenever a prisoner except for
the counter enters the cell for his second time, he turns the light
on. If a prisoner other than the counter enters the cell and the light
is on, he leaves it on, and if he was supposed to turn it on that
time, he remembers to turn it on the next time. When the counter
enters the cell and the light is on, he turns it off and increments
the number $x$ that he is thinking of. When the counter is thinking of
$P$, he announces that every prisoner has visited the room at least once.
\item You do not know the initial state of the switch.\\
The same as in the original case, except the counter waits until $x =
P + 1$. 

\end{itemize}


\item Design a strategy that allows $P - 1$ of $P$ prisoners to be
  freed.\\
The first prisoner determines the parity of blue hats in front of
him. If it is even, he says ``Red'', otherwise he says ``Blue.'' Every other
prisoner then determines the parity of blue hats in front of
him. If the parity is even, he starts to think of the same color that
the first prisoner said, if it is odd, he thinks of the opposite
color. Whenever a prisoner hears an earlier prisoner say ``Blue,'' he
switches the color that he is thinking of to the opposite
color. Whenever it is a prisoner's turn to answer, he says the color
that he is thinking of. Then, every prisoner except for the first one
will always correctly guess the color of his hat.

\num{6}
\item Solve for $S_n$.
\begin{align*}
S_2 &= \frac 1 {1 - p + \frac p 2}\\
p &= 2 - \frac 2 S_2 \\
S_n &= \frac 1 {1 - (2 -  \frac 2 S_2) + \frac {2 - \frac 2 S_2} n}
 = \frac 1 {\frac 2 S_2 + \frac{2 - \frac 2 S_2} n - 1}
\end{align*}

\item When is the multiprocessor superior?\\
Plugging in the number of processors to Amdahl's law, the multiprocessor is superior when:
\begin{equation*}
\frac{1}{(1 - p) + \frac p {10}} = \frac 1 {10 - 9p} > 5
\end{equation*}\\
so the multiprocessor solution is superior for applications that are
at least $88.\overline{8}\%$ parallel. 

\num{10}
\item Does/Is the \texttt{Flaky} lock\dots 
\begin{enumerate}
\item satisfy mutual exclusion?\\
Yes. For any two threads A and B, note that:
\begin{align*}
W_A(turn=A) \rightarrow R_A(busy==false) \rightarrow W_A(busy = true) \rightarrow R_A(turn == A) \rightarrow CS_A \\
W_B(turn=B) \rightarrow R_B(busy==false) \rightarrow W_B(busy = true) \rightarrow R_B(turn == B) \rightarrow CS_B
\end{align*}

Suppose, for contradiction, that there exist some executions $i, j$
such that $CS^i_A \not \rightarrow CS^j_B$ and $CS^i_B \not\rightarrow CS^j_A$. Suppose,
without loss of generality, that A is about to enter its critical
section. Then it must be the case that $R_A(turn == A)$ has just
occurred. If any other thread B is about to enter its critical section,
then it must be the case that:

\begin{align*}
W_A(turn=A) \rightarrow R_A(turn == A) \rightarrow R_B(turn == B)
\end{align*}

\contradiction
\item deadlock free?\\
No. With any number of threads, the following sequence of events can
occur:
\begin{itemize}
\item The lock is initialized, and \texttt{busy = false}.
\item Each thread executes line 5-9, i.e. each thread leaves the inner
  loop before \texttt{busy = true}.
\item Each thread except for the thread whose ID is currently stored
  in \texttt{turn} returns to the top of the loop and blocks at line 9.
\item The remaining thread returns to the top of the loop and blocks
  at line 9.
\end{itemize}

The algorithm will now no longer advance.
\item starvation free?\\
No, because deadlock implies starvation.

\end{enumerate}

\num{13}
\item Modify \texttt{Filter} to be an $\ell$-exclusion algorithm.
\begin{lstlisting}[language=Java]
class Filter implements Lock {
	int[] level;
	int[] victim;
	public Filter(int n, int L) {
		level = new int[n];
		victim = new int[n - L];
		for (int i = 0; i < n; i++) {
			level[i] = 0;
		}
	}
	public void lock() {
		int me = ThreadID.get();
		for (int i = 1; i < n - L; i++) { 
			int num_conflicts;
			level[me] = i;
			victim[i] = me;

			while (victim[i] == me) {
				num_conflicts = 0;
				for (int k = 0; k < n; k++){
					if((k != me) && (level[k] >= i)){
						if(++num_conflicts >= L) return;
					}
				}
			}
		}
	}
	public void unlock() {
		int me = ThreadID.get();
		level[me] = 0;
	}
}
\end{lstlisting}

\item \texttt{FastPath} neither provides mutual exclusion nor is
  starvation free. 
\begin{itemize}
\item \texttt{FastPath} does not provide mutual exclusion because if
  two threads read line 8 at the same time, both will enter their
  critical sections: one by going through \texttt{lock.lock} and the
  other by exiting \texttt{FastPath}'s \texttt{lock} method.
\item \texttt{FastPath} is not starvation free, because it could be
  the case that every time one thread gets a chance to execute an
  instruction, it is stuck on line 8, and other threads have set
  \texttt{y} to a value other than -1. 
\end{itemize}

\item Prove that\dots
\begin{itemize}
\item At most one thread gets the value \texttt{STOP}.\\
Suppose some thread A just executed line 14 and it is about to return
\texttt{STOP}. Then it must be the case that \texttt{last == i} and
\texttt{goRight == true}. All threads which have not yet executed line
11 will return \texttt{RIGHT}. All other threads which have executed
line 11 return \texttt{DOWN}, since \texttt{last == i} and no thread
can change the value of \texttt{i} if it has already executed line 11.
\item At most $n -1$ threads return \texttt{DOWN}.\\
If some thread has already returned \texttt{STOP}, then no more than
$n-1$ threads will return \texttt{DOWN}, since each thread returns
only once. If no thread has yet returned \texttt{STOP}, then every
thread must have already executed lines 11-13. One of these threads
will be the thread such that \texttt{last == i}, and this thread will
return \texttt{STOP}. Therefore no more than $n-1$ threads will return
\texttt{DOWN}.
\item At most $n -1$ threads return \texttt{RIGHT}. \\
If any thread returns \texttt{RIGHT}, then it must be that
\texttt{goRight == true}, and so at least one thread must have
executed line 13, and at least one thread cannot return
\texttt{RIGHT}. Therefore at most $n-1$ threads return \texttt{RIGHT}.
\end{itemize}
\end{enumerate}
\end{document}
