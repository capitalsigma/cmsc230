\documentclass{article}
\usepackage{listings}

\lstset{language=C}

\newcommand{\code}[1]{\begin{lstlisting}[language=C] #1 \end{lstlisting}}
  
\begin{document}
\title{Homework 1 - Design Writeup}
\author{Patrick Collins}
\date{\today}
\maketitle

\section{Sequential} 

\subsection{InputScanner}
Properties:
\begin{itemize}
\item \texttt{public int verticies}
\item \texttt{private int width}
\item \texttt{private int height}
\item \texttt{private int max\_dist}
\item \texttt{private int inf\_dist}
\end{itemize}
Invariants: \texttt{height <= width == verticies}

Methods:
\begin{itemize}
\item \texttt{public InputScanner* init(InputScanner *self)}\\
Given a valid pointer to an InputScanner, initialize the object.

\item \texttt{public Graph* read\_file(InputScanner *self, file *f)}\\
Read a text file representing a graph in distance-matrix
format. Return a new Graph object with ``unknown'' cells initialized
to \texttt{inf\_dist} and all ``known'' \texttt{cells <= max\_dist}.

Preconditions: \texttt{f} is a valid file handle.\\
Postconditions: \texttt{self.vertices == self.width == self.height}

\item \texttt{private int* read\_line(InputScanner *self, file *f)}\\
Read a line of the input file and return a list of \texttt{int}s
representing a row of the new array. If EOF has been reached, return
\texttt{NULL}. If \texttt{self->verticies} has not yet been set, then set it to
the length of the array to be returned. 

Preconditions: \texttt{f} is a valid file handle.\\
Postconditions: \texttt{(rv == NULL) || (length(rv) == self->verticies) ||
  (self->vertices == 0)}

\end{itemize}


% Methods:
% \begin{itemize}
% \item \texttt{graph make\_graph(int n)}
% Signature: \texttt(int \rightarrow graph)
% Description: Given an integer \texttt{n}, return an 

% \subsection{DistArray}
% Methods:



\end{document}
